\documentclass{article}
\usepackage[utf8]{inputenc}
\usepackage{amsthm,amsmath,amsfonts}
\usepackage{todonotes,xcolor,hyperref,cleveref}
\usepackage{nicefrac}

\newcommand{\cdt}{\mathrm{CDT}}
\newcommand{\prot}{\mathrm{PROT}}
\newcommand{\kw}{\mathrm{KW}}
\newcommand{\xvars}{x_{i_1}, \ldots , x_{i_s}}

\newcommand{\calP}{\mathcal{P}} 
\newcommand{\calQ}{\mathcal{Q}}

\newcommand{\depth}{depth}

\newcommand{\tulasi}[1]{{\color{teal} [{\bf Tulasi:} #1]}}

\title{Criticality of monotone De Morgan formulas}
\author{Tulasimohan Molli}
\date{October 2020}

%macros
%names
\newcommand{\demorgan}{De-Morgan\ }
\newcommand{\hastad}{H\aa stad}


%functions%
\newcommand{\sym}{\mathrm{SYM}}
\newcommand{\thr}{\mathrm{LTF}}
\newcommand{\maj}{\mathrm{MAJ}}
\newcommand{\IN}{\mathrm{IN}}
\newcommand{\AND}{\mathrm{AND}}
\newcommand{\OR}{\mathrm{OR}}
\newcommand{\MOD}{\mathrm{MOD}}
\newcommand{\xor}{\oplus}

%shortcuts%
\newcommand{\inner}[2]{\left\langle #1,#2 \right\rangle}
\newcommand{\setcond}[2]{\left\{#1\: \middle|\: #2\right\}}
\newcommand{\abs}[1]{\left| #1 \right|}

%probability
\newcommand{\prob}[2]{\Pr_{#1}\left[#2\right]}
\newcommand{\E}[2]{\mathbb{E}_{#1}\left[#2\right]}
\newcommand{\Ex}[2]{\underset{#1}{\mathbb{E}} \left[ #2 \right]}


%boolean functions and cubes
\newcommand{\cube}[1]{\{0,1\}^{#1}}
\newcommand{\fcube}[1]{\{-1,1\}^{#1}}
\newcommand{\bool}[1]{\cube{#1} \rightarrow \cube{}}
\newcommand{\fbool}[1]{\fcube{#1} \rightarrow \fcube{}}

%Brackets%
\newcommand{\bra}[1]{\left\{#1\right\}}
\newcommand{\sbra}[1]{\left[#1\right]}
\newcommand{\abra}[1]{\left\langle#1\right\rangle}
\newcommand{\cbra}[1]{\left(#1\right)}

%Symbols%
\newcommand{\IF}{\mathrm{if}}
\newcommand{\F}{\mathbb{F}}
\newcommand{\p}{\mathcal{P}}
\newcommand{\Z}{\mathbb{Z}}
\newcommand{\R}{\mathbb{R}}

%asymptotic notation
\newcommand{\bigo}[1]{O\left(#1\right)}
\newcommand{\smallo}[1]{o\left(#1\right)}

\newcommand{\ie}{i.e.}

%%environments
%\theoremstyle{plain}
\newtheorem{theorem}{Theorem}
%\newtheorem*{theorem*}{Theorem}
\newtheorem{corollary}[theorem]{Corollary}
\newtheorem{lemma}[theorem]{Lemma}
%\newtheorem*{lemma*}{Lemma}
\newtheorem{observation}[theorem]{Observation}
\newtheorem{proposition}[theorem]{Proposition}
\newtheorem{definition}[theorem]{Definition}
\newtheorem{claim}[theorem]{Claim}
%\newtheorem*{claim*}{Claim}
\newtheorem{fact}[theorem]{Fact}
\newtheorem{assumption}[theorem]{Assumption}
\newtheorem{remark}[theorem]{Remark}
\newtheorem{conjecture}{Conjecture}
%\newtheorem*{conjecture*}{Conjecture}
\newtheorem{question}{Question}
%\newtheorem*{question*}{Question}
%\newtheorem*{answer*}{Answer}
%%commands%




\begin{document}

\maketitle    
    \begin{theorem}
    \label{thm:main theorem}
    Let $s$ be a natural number. Let $F$ be a monotone De-Morgan formula with $L(F)$ leaves. Then 
    \[
        \prob{\rho}{\cdt(F,\rho) \text{ has \depth\ } \geq s} \leq (16p\sqrt{L(F)})^s
    \]
\end{theorem}

\section{Introduction}
\subsection{Decision trees}
    Query algorithms and decision tree. 
    A decision tree is a binary tree which is a representation of a non-adaptive query algorithm. 

\begin{definition}
    A decision tree is a binary tree where the nodes are labelled by variables, edges are labelled by 0 or 1 and leaves are labelled by 0 or 1.    
\end{definition}
    We refer to paths in a decision tree as \emph{branches}.
    Each branch corresponds to an ordered partial assignment of input variables.
    We use $\abra{x_{i_1} \to a_1, \ldots , x_{i_s} \to a_s}$ to represent the branch whose nodes are labelled by $x_{i_1}, \ldots ,x_{i_s}$ and the edges are labelled by $a_1, \ldots , a_s$ in that order.     
%    We refer to concatination of branches in a tree as a \emph{graft}.

    Decision trees compute boolean functions in a natural way. 
    A decision trees $T$ computes a boolean function $F$ is at all the root to leaf branches the function becomes a constant and the value of the function is given by the label at the leaf.  
    
%    If a branch sets the function to 0, we call it a 0-branch certificate (similarly for 1-branch certificate).
%    If a concatination of branches of the decision tree sets the function to 0, we call it a 0-graft certificate (similarly for 1-graft certificate). 
       
\section{Canonical Decision tree:}    
    Given a De-Morgan formula $F$ and a restriction $\rho$, we define the canonical decision tree of $F$ w.r.t. the restriction $\rho$ denoted by $\cdt(F,\rho)$ as follows
    \begin{definition}[$\cdt(F,\rho)$:]
        The canonical decision tree is defined inductively as follows:
        \begin{itemize}
            \item If $F|_\rho$ is a constant 0 or 1, then $\cdt(F,\rho)$ is a single node labelled appropriately by that constant.
            \item If $F$ is a literal $x_i$ (or $\lnot x_i$), then $\cdt(F,\rho)$ consists of a node labelled by $x_i$  with two children which are labelled by 0 and 1.
            \item If $F = F_1 \vee F_2$:
                \begin{itemize}
                    \item If $F_1|_\rho$ is non-constant,
                    %
                    $\cdt(F,\rho)$ is the set of ordered restrictions $\abra{v_1 \to a_1, \dots ,v_u \to a_u}$ of length $u \geq 1$ such that 
%                    construct $\cdt(F_1,\rho)$. For each leaf $\alpha$ in $\cdt(F_1,\rho)$ labelled $0$, attach $\cdt(F_2, \rho \circ \alpha)$.     
                    there exists $t \in [u]$ and $b \in \cube{t}$satisfying 
                    \begin{itemize}
                        \item $\abra{v_1 \to b_1,\dots ,v_t \to b_t} \in \cdt(F_1,\rho)$,
                        \item $\abra{v_{t+1} \to a_{t+1}, \dots , v_u \to a_u} \in \cdt(F_2, \rho\circ \abra{v_1 \to a_1, \dots , v_t\to a_t})$, and 
                        \item for all $i \in [t]$,
                        $$b_i = \begin{cases}
                                a_i & if F_1|_{\rho}|_{\abra{v_1 \to b_1, \dots v_{i-1} \to b_{i-1},v_i \to a_i}} \not\equiv 0\\
                                1-a_i & if F_1|_{\rho}|_{\abra{v_1 \to b_1, \dots v_{i-1} \to b_{i-1},v_i \to a_i}} \equiv 0,
                        \end{cases}
                        $$
                    \end{itemize}
                    \item If $F_1|_\rho \equiv 0$ output $\cdt(F_2,\rho)$.  
                \end{itemize}             
            \item In the case of $F = F_1 \wedge F_2$,  $\cdt(F,\rho)$ is defined similar to the $F= F_1 \vee F_2$ case with the roles of 0 and 1 interchanged. 
        \end{itemize}    
    \end{definition}   


    \begin{claim}[Unpacking]
    \label{claim:unpacking}
        Let $F = F_1 \vee F_2$ be a monotone De Morgan formula.
        Let $u \geq 1$, $a \in \cube{u}$ and  $(x_{i_1} \to a_1, \ldots , x_{i_u} \to a_u)$ be an ordered restriction.
        If $\cdt(F,\rho)$ has the path $(x_{i_1} \to a_1, \ldots , x_{i_u} \to a_u)$, then either       
%        of length $u \geq 1$ 
%        then there exist variables
%        $v_1,\ldots , v_u$ and an assignment such that such that
%        either         
    \begin{itemize}
        \item $F_1|_{\rho} \equiv 0$ and $\cdt(F_2,\rho)$ has the path $(x_{i_1} \to a_1, \ldots , x_{i_u} \to a_u)$,     
        or 
        \item there exists $t \in [u]$, $b \in \cube{t}$ such that 
        \begin{itemize}
            \item $F_1|_{\rho \circ (v_1 \to a_1, \dots , v_t \to a_t)} = 0$,
            \item $F_1|_{\rho \circ (v_1 \to b_1, \dots , v_t \to b_t)} = 1$,
            \item $(v_1 \to b_1, \dots , v_t \to b_t)$ is a path in $\cdt(F_1,\rho)$,
            \item $(v_{t+1} \to a_{t+1}, \dots , v_u \to a_u)$ is a path in $\cdt(F,\rho \circ (v_1 \to a_1, \dots , v_t \to a_t))$.
        \end{itemize}
    \end{itemize}
    \tulasi{need to explain why F1 is 1 on the b assignment} 
    \end{claim}

\section{Structural Lemma}



\begin{lemma}
    \label{lem:structural lemma}
    Let $F$ be a monotone De Morgan formula and $\rho$ be a restriction. Let $u$ be a natural number such that $\cdt(F,\rho)$ has a path of length $u$ given by $(x_{i_1} \to a_1, \dots , x_{i_u} \to a_u)$. Then, there exist
    $u$ root to leaf paths $p_1, \ldots , p_u$ in the formula tree $F$ whose leaves are labelled by $\xvars$ respectively. 
    
    Assignments $\rho_0$ and $\rho_1$ such that 
    $\rho_0 := \rho \circ \abra{i_1 \to 0, \dots , i_u \to 0}$
    and
    $\rho_1 := \rho \circ \abra{i_1 \to 1, \dots , i_u \to 1}$
    there exist $s$ variables $x_{i_1},\cdots ,x_{i_u}$ and leaves $\ell_1, \ldots, \ell_s$ such that 
    \begin{itemize}
        \item $G|_{\rho_0} \equiv 0$ and $G_{\rho_1} \equiv 1$ for all nodes $G$ in $p_1 \cup \dots \cup p_u$. 
    \end{itemize}
\end{lemma}    

%unpacking => Strucutal lemma
\begin{proof}
    The proof is by induction on the size of the formula $F$, 
    Consider the case when $F = F_1 \vee F_2$. The case when $F = F_1 \wedge F_2$ can be handled similarly.\tulasi{elaborate}
    Suppose $\cdt(F,\rho)$ has of path of length $u$ given by $(i_1 \to a_1, \ldots ,i_u \to a_u)$, by~\cref{claim:unpacking}, 
    \begin{itemize}
        \item[Case 1:] If $F_1|_\rho \equiv 0$ and $\cdt(F_2,\rho)$ has the path $(i_1 \to a_1, \dots ,i_u \to a_u)$.\\
        Since $F_2$ is a smaller formula than $F$, by induction hypothesis. $F_2$ has $u$ paths which satisfy the required conditions and we are done. 
        \item[Case 2:] there exists $t \in [u]$ and $b \in \cube{t}$ such that 
        $(i_1 \to a_1, \dots , i_t \to a_t)$ is a path in $\cdt(F_1,\rho)$ and $(i_{t+1} \to a_{t+1}, \dots ,i_u \to a_u)$ is a path in $\cdt(F_2, \rho \circ (i_1 \to a_1, \dots , i_t \to a_t))$.
        
        Therefore, by induction hypothesis there exists $t$ paths $p_{1,1}, \dots , p_{1,t}$ in $F_1$ labelled by $x_{i_1}, \dots , x_{i_t}$ and $u-t$ paths in $p_{2,1}, \dots , p_{2,u-t}$ in $F_2$ labelled $x_{i_{t+1}}, \dots , x_{i_u}$ which satisfy the required properties.
        
        Let $p_1, \dots , p_u$ be paths in $F$ which are obtained by extending each of the above paths with the root node of $F$.
        
        Let's check $p_1, \dots , p_u$ satisfy the required properties.
        
        Let $G \in \calP := p_1 \cup \dots \cup p_u$. We have the following three cases
        
        \begin{itemize}
            \item [$\bf{G \in F_1}$:] let $\rho^1_0 := \rho \circ (i_1 \to 0, \dots , i_t \to 0)$ and $\rho^1_1 := \rho \circ (i_1 \to 1, \dots , i_t \to 1)$. 
            By induction hypothesis, 
            for all $G \in F_1 \cap \calP$, $G|_{\rho^1_0} \equiv 0$ and 
            $G|_{\rho^1_1} \equiv 1$.
            Since $\rho_0$ and $\rho_1$ are extensions of $\rho^1_0$ and $\rho^1_1$ respectively, 
            $G|_{\rho_0} \equiv 0$ and $G|_{\rho_1} \equiv 1$.
            \item [$\bf{G \in F_2}$:]
                let $\rho^2_0 := \rho \circ (i_1 \to a_1, \dots , i_t \to a_t, i_{t+1} \to 0, \dots ,i_u \to 0)$ and $\rho^2_1 := \rho \circ (i_1 \to a_1, \dots , i_t \to a_t, i_{t+1 }\to 1, \dots i_u \to 1)$. 
                By induction hypothesis, 
            for all $G \in F_2 \cap \calP$, $G|_{\rho^2_0} \equiv 0$ and $G|_{\rho^2_1} \equiv 1$.
            Since $G$ is a monotone formula and $\rho_0 \leq \rho^2_0$ and $\rho_1 \geq \rho^2_1$, therefore
            $G|_{\rho_0} \equiv 0$ and $G|_{\rho_1} \equiv 1$.
            \item [$\bf{G = F}$:]
                
        \end{itemize}
        


    \end{itemize}
    
    
\end{proof}



\section{KW game and a Protocol}
    \paragraph{Game:} 
    Given a boolean function $F$, we consider a variant of the Karchmer-Wigderson game denoted by $\kw_F$. The game involves two players Alice and Bob.  
    Alice gets an input $x \in F^{-1}(0)$ and Bob gets $y \in F^{-1}(1)$. Their goal is output the indices of all coordinates where $x$ and $y$ differ.
    
\subsection{Protocol}
    \begin{lemma}
        Let $F$ be a monotone De-Morgan formula with $L(F)$ leaves.  
    \end{lemma}

%    Alice gets the input $\rho_0 = \rho \circ \abra{x_{i_1} \to b_1, \ldots x_{i_s} \to b_s}$ and Bob gets the input $\rho_1 = \rho \circ \abra{x_{i_1} \to \bar{b_1}, \ldots , x_{i_s} \to \bar{b_s}}$.
    Alice gets the input $\rho_0 = \rho \circ \abra{x_{i_1} \to 0, \ldots x_{i_s} \to 0}$ and Bob gets the input $\rho_1 = \rho \circ \abra{x_{i_1} \to 1, \ldots , x_{i_s} \to 1}$.
    Alice and Bob go through all possible runs of the KW protocol. 

%%%%%%%
    % The protocol proceeds in $s$ rounds. For $k \in [s]$, in the $k^{th}$ round, Alice and Bob recover the $i^{th}$ variable in the long path in the following way: 
    
    % They run KW protocol by picking the left child when ever they have a choice on the inputs $\rho\circ \abra{x_{i_1} \to a_1, \ldots x_{i_{k-1}} \to a_{k-1}, x_{i_k} \to b_k, \ldots , x_{i_{s}} \to b_{s}}$ and $\rho\circ \abra{x_{i_1} \to a_1, \ldots x_{i_{k-1}} \to a_{k-1}, x_{i_k} \to \bar{b_k}, \ldots , x_{i_{s}} \to \bar{b_{s}}}$. 
%%%%%%%    
    

    % Let $\prot(F,\rho_0,\rho_1)$ denote the protocol for $F$ when Alice and Bob get inputs $\rho_0$ and $\rho_1$ respectively. 
    
    % They run the following protocol $\prot(F,\rho_0,\rho_1)$

    % $\prot(F,\rho_0,\rho_1)$:\\
    %     If $F(\rho_0) = F(\rho_1):$ \\
    %         output $\bra{}$, 

    %     else if $F$ is a literal $x_i$ or $\lnot x_i$:\\
    %         output $\bra{i}$
            
    %     else if $(0= F_1(\rho_0) \neq F_1(\rho_0) = 1)$:\\
    %         output $\prot(F_1,\rho_0,\rho_1) \cup\prot(F_2,\rho_0,\rho_1)$,
        
    %     else return $\bot$.     
        
    % Let $\rho_0 := \rho \circ \alpha_0$  and $\rho_1 := \rho \circ \alpha _1$.       

\begin{claim}
    \label{claim:no. of transcripts}
    The number of transcripts of the above protocol is at most $L(F)^s$, and in fact at most $\binom{L(f)}{s}$.
\end{claim}

\newcommand{\kwleft}{\text{KWleft}}
          
\section{Proof of Theorem~\ref{thm:main theorem}}
\begin{proof}
    Consider the event that $\cdt(F,\rho)$ has a path of length $s$. By Lemma~\ref{lem:structural lemma}, there exist $s$ variables $\xvars$, and two assignments $a,b \in \cube{s}$ and a transcript $\pi$ of the protocol $\kwleft$ such that $\prot(F,a,\rho \circ \alpha_k, \rho \circ \beta_k) = \pi$. 
    \begin{align*}
        &\prob{\rho}{\cdt(F,\rho) \text{ has \depth } s}\\
            &\leq \sum_{\xvars} \sum_{a \in \cube{s}} \sum_{b \in \cube{s}} \sum_{\pi} \prob{\rho}{\prot(F,a,\rho_0,\rho_1) = \pi}\\
        & \text{ where }\rho_0 = \rho \circ \abra{x_{i_1} \to b_1, \ldots , x_{i_s} \to b_s} \text{ and }
        \rho_1 = \rho \circ \abra{x_{i_1} \to \bar{b_1}, \ldots , x_{i_s} \to \bar{b_s}}\\        
        &\leq 2^{2s} \max_{a^*,b^*}\sum_{\xvars} \sum_{\pi} \sum_{\rho\colon \prot(F,\rho_0,\rho_1) = \pi} \prob{}{\rho}\\
        &= \cbra{\frac{8p}{1-p}}^s \cdot \sum_{\xvars} \sum_{\pi} \sum_{\rho\colon \prot(F,\rho_0,\rho_1) = \pi} \sqrt{\prob{}{\rho_0}} \sqrt{\prob{}{\rho_1}}\\
        &= (16p)^s \cdot \sum_{\xvars} \sum_{\pi}  \sqrt{\sum_{\rho:\prot(F,\rho_0,\rho_1) = \pi} \prob{}{\rho_0}} \sqrt{\sum_{\rho:\prot(F,\rho_0,\rho_1) = \pi} \prob{}{\rho_1}} \tag*{since p $\leq$ 1/2.}\\
        &= (16p)^s \sum_{\xvars} \sum_{\pi} \sqrt{\prob{}{A_{\pi}}} \sqrt{\prob{}{B_{\pi}}}\\
        &\leq (16p)^s   \sqrt{\sum_{\xvars} \sum_{\pi} 1} \sqrt{\sum_{\xvars} \sum_{\pi} \prob{}{A_\pi}\prob{}{B_\pi}}\\
        &\leq (16p)^s\sqrt{\# \pi} \tag{by Claim~\ref{claim:sum of rectangles}}\\
        &\leq (16p)^s \sqrt{L(F)^s} \tag{by Claim~\ref{claim:no. of transcripts}}
    \end{align*}
    where $\prob{}{A_\pi} := \sum_{\rho:\prot(F,\rho_0,\rho_1) = \pi} \prob{}{\rho_0}$ and  $\prob{}{B_\pi} := \sum_{\rho:\prot(F,\rho_0,\rho_1) = \pi} \prob{}{\rho_1}$.
\end{proof}

    \begin{claim}
    \label{claim:sum of rectangles}
        $$\sum_{\xvars} \sum_{\pi} \prob{}{A_\pi}\prob{}{B_\pi} \leq 1$$
    \end{claim}
%\newcommand{\maj}{Maj}
\newcommand{\corr}{Corr}
\newcommand{\f}{\maj}
\newcommand{\g}{C}

\section{Correlation of Majority with monotone De-Morgan formulas}
    \begin{theorem}
    \label{thm:corr with maj}
        Let $\maj$ be the majority function on $n$ bits. 
        Let $C$ be any De-Morgan formula with $L$ leaves. Then,
        $$\prob{x}{\maj(x) = C(x)} \leq \frac12 + $$
    \end{theorem}

    \begin{proof}
        Let $E_1$ be the event that $\cdt(F,\rho)$ has depth $\leq \log n$, $E_2$ be the event that $|stars(r)| \geq pn/2$, $E_3$ be the event that $bias(\rho) \leq \sqrt{n \log n}$.
        \begin{align*}
            \prob{x}{\f(x),\g(x)} &
                        = \prob{\rho \sim R_p}{\prob{z \sim \fcube{stars(\rho)}}{\f|_{\rho}(z) = \g|_{\rho}(z)}} \\
                        &= \prob{\rho \sim R_p}{\prob{z \sim \fcube{stars(\rho)}}
                        {\f|_\rho(z) = \g|_\rho(z) \wedge (E_1 \wedge E_2 \wedge E_3(\rho))}}\\
                        &\hspace{10pt}+ \prob{\rho \sim R_p}{ \prob{z \sim \fcube{stars(\rho)}}
                        {\f(z)|_\rho = \g|_\rho(z) \wedge  (\bar{E_1} \vee \bar{E_2} \vee \bar{E_3}(\rho))}}\\
                        &\leq  \prob{z \sim \fcube{stars(\rho)}}{\f|_\rho(z)\cdot \g|_\rho(z)| E_1 \wedge E_2 \wedge E_3} + \prob{\rho}{\bar{E_1} \vee \bar{E_2} \vee \bar{E_3}}
        \end{align*}
    \end{proof}
    
    
    
\tulasi{Correlation of Majority with small depth decision trees.}
\begin{claim}
    Let $T$ be a decision tree of depth $d$. 
        $$Corr(T,\maj_n) \leq \frac{1}{2^{n-d}}\sum_{j=0}^{n/2 -d} \binom{n-d}{j}$$
\end{claim}
\tulasi{bound chrenoff events.}

By Chernoff bounds, $\prob{}{bias(\rho) \geq \sqrt{n \log n}} \leq \frac1n$. 

By Chernoff bounds, $\prob{}{stars(\rho) \leq pn/2} \leq e^{-pn/6}$.

By Switching lemma for monotone De-Morgan formulas 
$\prob{}{E_1} \leq (p\sqrt{L})^{\log n} \leq \frac1n$.

    $$ \frac12 - \frac{\log {pn}}{\sqrt{pn}}\leq \prob{\rho,z}{\maj|_\rho(z) = C|_\rho(z)|E_1 \wedge E_2 \wedge E_3} \leq \frac12 + \frac{\log pn}{\sqrt{pn}}$$
\end{document}
